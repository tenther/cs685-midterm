\documentclass[letter,8pt]{article}
\usepackage[margin=0.5in]{geometry}
\usepackage[pdftex]{graphicx}
\usepackage{color}
\usepackage{titlesec}
\usepackage{enumitem}
\usepackage{bm}
\usepackage{amsmath}
\usepackage{textcomp}
\usepackage{hyperref}
\titleformat{\section}
  {\normalfont\sffamily\Large\bfseries}
  {\thesection}{1em}{}

\begin{document}
\title{Computer Science 685 Midterm}
\author{Paul McKerley (G00616949)}
\fontfamily{phv}
\selectfont
\frenchspacing

\maketitle 

\begin{enumerate} 

\item If I understand the question correctly, it is to define the
  forward kinematics of the point at the end of $L2$. Since this is
  the position of the gripper, I will refer to it as $G$. I interpret
  the diagram to mean that $L1$ is a fixed bar, which slides up and
  down $y1$, causing $d1$ to vary. Also, $\theta_2$ can change,
  causing $L2$, which is also fixed, to swing around.

  So the position of $G$ in the second coordinate frame is simply how
  far it is translated from the end of $L1$, which is

\[
    G_2 = 
    \begin{bmatrix}
        L2\\
        0
    \end{bmatrix} 
\]

  Now in the first coordinate frame, $G$ is rotated by $\theta_2$ and
  translated by $L1$ on the $x$ axis and $d1$ on the $y$ axis. So

  \[
  G_1 = \begin{bmatrix}
    L1\\
    d1
  \end{bmatrix} + R(\theta_2) \cdot     \begin{bmatrix}
        L2\\
        0
    \end{bmatrix}  = \begin{bmatrix}
    L1\\
    d1
  \end{bmatrix} + \begin{bmatrix}
    L2 cos(\theta_2)\\
    L2 sin(\theta_2)
  \end{bmatrix} = \begin{bmatrix}
    L1 + L2 cos(\theta_2)\\
    d1 + L2 sin(\theta_2)
  \end{bmatrix}
  \]

As mentioned above, the kinematic parameters are $L1$, $L2$, and
implicitly the angles of $d1$ with respect to $x1$, and $L1$ with
respect to to $d1$. The joint variables are $d1$ and $\theta_2$.

\item 

\item 

\item 

\end{enumerate}

\end{document}
\endinput
