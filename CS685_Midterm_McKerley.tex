\documentclass[letter,8pt]{article}
\usepackage[margin=0.5in]{geometry}
\usepackage[pdftex]{graphicx}
\usepackage{color}
\usepackage{titlesec}
\usepackage{enumitem}
\usepackage{bm}
\usepackage{amsmath}
\usepackage{textcomp}
\usepackage{hyperref}
\titleformat{\section}
  {\normalfont\sffamily\Large\bfseries}
  {\thesection}{1em}{}

\begin{document}
\title{Computer Science 685 Midterm}
\author{Paul McKerley (G00616949)}
\fontfamily{phv}
\selectfont
\frenchspacing

\maketitle 

\begin{enumerate} 

\item If I understand the question correctly, it is to define the
  forward kinematics of the point at the end of $L2$. Since this is
  the position of the gripper, I will refer to it as $G$. I interpret
  the diagram to mean that $L1$ is a fixed bar, which slides up and
  down $y1$, causing $d1$ to vary. Also, $\theta_2$ can change,
  causing $L2$, which is also fixed, to swing around.

  So the position of $G$ in the second coordinate frame is simply how
  far it is translated from the end of $L1$, which is

\[
    G_2 = 
    \begin{bmatrix}
        L2\\
        0
    \end{bmatrix} 
\]

  Now in the first coordinate frame, $G$ is rotated by $\theta_2$ and
  translated by $L1$ on the $x$ axis and $d1$ on the $y$ axis. So

  \[
  G_1 = \begin{bmatrix}
    L1\\
    d1
  \end{bmatrix} + R(\theta_2) \cdot     \begin{bmatrix}
        L2\\
        0
    \end{bmatrix}  = \begin{bmatrix}
    L1\\
    d1
  \end{bmatrix} + \begin{bmatrix}
    L2 cos(\theta_2)\\
    L2 sin(\theta_2)
  \end{bmatrix} = \begin{bmatrix}
    L1 + L2 cos(\theta_2)\\
    d1 + L2 sin(\theta_2)
  \end{bmatrix}
  \]

As mentioned above, the kinematic parameters are $L1$, $L2$, and
implicitly the angles of $d1$ with respect to $x1$, and $L1$ with
respect to to $d1$. The joint variables are $d1$ and $\theta_2$.

\item
  \begin{enumerate}[label=\alph*)]
    \item The function needs $U_r$ to be expanded to include repulsion from robot $R2$ in position $x_2$. So it would look something like\\
      $U(x_1) = k_1||x_1 - x_g||^2 + k_2\frac{1}{||x_1 - x_o||^2} + k_3\frac{1}{||x_2 - x_1||^2}$\\
      with the repulsion of the $R2$ possibly having a different constant $k_3$, since a moving obstacle might require speedier avoidance.

    \item The control function will be to figure out the forces applied to $R1$ at that position. 
      We need to define distances of influence for the obstacle and the other robot, so, we will define $q_o$ and $q_2$
      to be the minimum distances from the obstacle and other robot, respectively, before they have a repulsive effect.
      \[
      F(x_1) = -k_1 \cdot ( x_1 - x_g) -
      \begin{cases}
        (k_2(\frac{1}{x_1} - \frac{1}{x_o})), & \text{if} x=1 \\
        0, & \text{otherwise}
       \end{cases}
      \]
    \item The repulsive and attractive forces at a certain location can cancel each other out perfectly, so that 
      the net force applied at that location is $0$; ie, the robot encounters a local mininum.
      When this happens the robot could adopt a number of strategies. When if finds itself stuck, it could go on a 
      random walk and hope to leave the local minimum. It could use a strategy like one of the Bug algorithms. 
      For example, the Bug2 algorithm tells the robot that when it can't move, it should remember its bearing
      to the goal, and circle around the obstacle until it hits that bearing line again, at which point it should
      continue toward the goal.
  \end{enumerate}
  
\item I'm assuming that the $d_x^2$ and $d_z^2$ variables are constant for a given run.
  It seems that the variance of our state converges quickly to a stable value. As $d_x^2$ and $d_z^2$
  approach zero, the point of stabalization becomes closer and closer to zero. I interpret this to mean
  that as the variances in the perturbation and sensor reading becomes more and more accurate (ie, $d_x^2$ and $d_z^2$ approach zero),
  then the robots estimate of its position likewise becomes more accurate.

\item 

\end{enumerate}

\end{document}
\endinput
